\documentclass[UTF8,12pt,a4paper]{report}
    \usepackage{graphicx}
    \usepackage{ctex}
    \usepackage{indentfirst}

    \usepackage{CJK}
    \usepackage{fancyhdr}
    \usepackage{graphicx}
    \usepackage{titlesec}
    \usepackage{titletoc}
    \usepackage{listings}
    \usepackage{appendix}
    \usepackage{bm, amsmath,amsfonts}
    \usepackage{multirow}
    \setlength{\parindent}{12pt}%自然段第一行的缩进量为12pt
    
    \setlength{\parskip}{10pt plus1ptminus1pt}
    
    %自然段之间的距离为10pt,并可在8pt到11pt之间变化
    
    \setlength{\baselineskip}{8pt plus2ptminus1pt}
    
    %行间距为8pt,并可在7pt到10pt之间变化
    
    \setlength{\textheight}{21truecm}%版面高为21厘米
    
    \setlength{\textwidth}{14.5truecm}%版面宽为14.5厘米
    \graphicspath{{pic/}}
    \renewcommand{\contentsname}{\zihao{3} 目\quad 录}
    \renewcommand{\abstractname}{\zihao{3} 摘\quad 要}
    %页眉页脚设置
    \pagestyle{fancy}
    \fancyhf{}
    \cfoot{\thepage}
    \rhead{\kaishu~XX课程设计~}
    
%目录页设置
\titlecontents{section}[0em]{\zihao{4}\bf }{\thecontentslabel\ }{}
{\hspace{.5em}\titlerule*[4pt]{$\cdot$}\contentspage}
\titlecontents{subsection}[2em]{\vspace{0.1\baselineskip}\zihao{-4}}{\thecontentslabel\ }{}
{\hspace{.5em}\titlerule*[4pt]{$\cdot$}\contentspage}
\titlecontents{subsubsection}[4em]{\vspace{0.1\baselineskip}\zihao{-4}}{\thecontentslabel\ }{}
{\hspace{.5em}\titlerule*[4pt]{$\cdot$}\contentspage}
%代码设置
\RequirePackage{listings}
\RequirePackage{xcolor}
\definecolor{dkgreen}{rgb}{0,0.6,0}
\definecolor{gray}{rgb}{0.5,0.5,0.5}
\definecolor{mauve}{rgb}{0.58,0,0.82}
\lstset{
	numbers=left,  
	frame=tb,
	aboveskip=3mm,
	belowskip=3mm,
	showstringspaces=false,
	columns=flexible,
	framerule=1pt,
	rulecolor=\color{gray!35},
	backgroundcolor=\color{gray!5},
	basicstyle={\ttfamily},
	numberstyle=\tiny\color{gray},
	keywordstyle=\color{blue},
	commentstyle=\color{dkgreen},
	stringstyle=\color{mauve},
	breaklines=true,
	breakatwhitespace=true,
	tabsize=3,
}
%------------------------------------------------------------------------
\begin{document}
    %%%%%%%%%%%%%%%%%%%%%%%%%%%%%%
    %% 封面部分
    %%%%%%%%%%%%%%%%%%%%%%%%%%%%%%
\begin{titlepage}
	\centering
	\includegraphics[width=0.9\textwidth]{wordart.jpg}\par
	\vspace{1cm}
			
	%% {\scshape\LARGE Harbin Institute of Technology \par}
	%% \vspace{1cm}
	%%{\kaishu\LARGE 概率论课程论文\par}
	%% \vspace{1.5cm}
	%{
	{ \huge 数据结构课程报告\par}
	\vspace{2cm}
	\fontsize{14pt}\baselineskip
		 \makebox[30mm]{设计题目}
		 \underline{\makebox[50mm][c]{ 题目}}\\%在这里修改成自己的题目
		 \vskip 0.2cm
		 \makebox[30mm]{学生姓名}
		 \underline{\makebox[50mm][c]{ 姓名}}\\
		 \vskip 0.2cm
		 \makebox[30mm]{学\qquad 号}
		 \underline{\makebox[50mm][c]{  2016214000}}\\
		 \vskip 0.2cm
		 \makebox[30mm]{专业班级}
		 \underline{\makebox[50mm][c]{ 班级}}\\
		 \vskip 0.2cm
		  \makebox[30mm]{指导教师}
		 \underline{\makebox[50mm][c]{ 教师}}\\
		 \vskip 2cm

	% Bottom of the page
	\date{2018/8/23}
		%{\large \today\par}
\end{titlepage}
%%%%%%%%%%%%%%%%%%%%%%%%%%%%%%%%%%%%%%%%%%%%%%%5%%%%%%%%%%%%%%%%%%%%%%%%%page1
\newpage
\section{第一节}
是测试文字\\这是测试文字这是测试文字这是测试文字这
\begin{tabular}{|c|c|} 
    a & b \\ 
    c & d\\ 
  \end{tabular} 
  \begin{tabular}{|c|c|} 
    \hline 
    a & b \\ 
    \hline 
    c & d\\ 
    \hline 
  \end{tabular} 

  \begin{center} 
    \begin{tabular}{|c|c|} 
      \hline 
      a & b \\ \hline 
      c & d\\ 
      \hline 
    \end{tabular} 
  \end{center} 
%%%%%%%%%%%%%%%%%%%%%%%%%%%%%%
%    代码块
	\subsection{程序代码}
	\begin{lstlisting}[language=C++,escapeinside=``]
	#include<iostream>
	using namespace std;
	int main
	{
		cout<<"Hello world!"<<endl;//`输出`
		return 0;
	}
	\end{lstlisting}
%%%%%%%%%%%%%%%%%%%%%%%%%%5
%%%%%%%%%%%%%%%%%%%%%%%%%%%%%%%%%%%%%%%%%%%%%%%%%%%%%%%%%%%%%%%%%%%%%%%%%%%%555page2
\newpage
\section{表格}
\begin{table}[h]
	\caption{排序算法对比}
	\centering
	\begin{tabular}{||c|c|c|c|c|c||}
		\hline
		\multirow{2}*{类别}   &\multirow{2}*{排序方法} &\multicolumn{3}{|c|}{时间复杂度} &\multirow{2}*{稳定性}\\
		\cline{3-5}
		& &平均情况&最好情况&最坏情况& \\
		\hline
		\multirow{2}*{插入排序}&直接插入&$O(n^2)$&$O(n)$&$O(n^2)$&稳定\\
		\cline{2-6}
		&Shell排序&$O(n^{1.3})$&$O(n)$&$O(n^2)$&不稳定\\
		\hline
		\multirow{2}*{选择排序}&直接选择&$O(n^2)$&$O(n^2)$&$O(n^2)$&不稳定\\
		\cline{2-6}
		&堆排序&$O(n\log_{2}n)$&$O(n\log_{2}n)$&$O(n\log_{2}n)$&不稳定\\
		\hline
		\multirow{2}*{交换排序}&冒泡排序&$O(n^2)$&$O(n)$&$O(n^2)$&稳定\\
		\cline{2-6}
		&快速排序&$O(n\log_{2}n)$&$O(n\log_{2}n)$&$O(n^2)$& 不稳定\\
		\hline
		\multicolumn{2}{||c|}{归并排序}&$O(n\log_{2}n)$&$O(n\log_{2}n)$&$O(n\log_{2}n)$&稳定\\
		\hline
		\multicolumn{2}{||c|}{基数排序}&$d(r+n)$&$d(n+rd)$&$d(r+n)$& 稳定\\
		\hline
	\end{tabular}
\end{table}
\end{document}