\documentclass[11pt]{article}
    %文件类说明

    %还可以选择的类是cctbook
    
    \setlength{\parindent}{12pt}%自然段第一行的缩进量为12pt
    
    \setlength{\parskip}{10pt plus1ptminus1pt}
    
    %自然段之间的距离为10pt,并可在8pt到11pt之间变化
    
    \setlength{\baselineskip}{8pt plus2ptminus1pt}
    
    %行间距为8pt,并可在7pt到10pt之间变化
    
    \setlength{\textheight}{21truecm}%版面高为21厘米
    
    \setlength{\textwidth}{14.5truecm}%版面宽为14.5厘米
    \usepackage[left=3cm, right=3cm, top=2.5cm, bottom=2.5cm]{geometry}
    \usepackage{leftidx}
    %\usepackage{CJKutf8}
    %\begin{CJK}{UTF8}{song}
    
    \begin{document}%正文开始
    
    \title{Thesis}%文章标题,双反斜杠\\表示换行
    
    \author{author\\Dept. of Math.}
    
    %作者名,单位,通信地址等,双反斜杠\\表示换行
    
    \date{2003/8/5}
    
    %文章写作日期,如果省略此行,计算机日期作为写作日期
    
    \maketitle%建立标题部分
    
    %文章的正文输入
    
    \begin{center}%参考文献的书写
    
    { 参考文献}
    
    \end{center}
    
    \vskip 0.1cm
    
    \def\hang{\hangindent\parindent}
    
    \def\textindent#1{\indent\llap{#1\enspace}\ignorespaces}
    
    \def\re{\par\hang\textindent}
    
    \re{[1]} Nordhaus E,Stewart B,WhiteA.On theMaximum Genus of a Graph.{\it J.combinatorial TheoryB},1971,11:258-267  
    
    \re{[2]} Skoviera M.The Maximum Genus of Graphsof Diameter Two.{\it Discrete Math}.1991, 87:175-180
    
    %\end{CJK}

    \end{document}%源文件的结束